\documentclass[10pt,a4paper]{article}
\usepackage[utf8]{inputenc}
\usepackage[english]{babel}
\usepackage[english]{isodate}
\usepackage[parfill]{parskip}
\usepackage{textcomp}

%----------------------------------------------------------------------------------------
%	BLANK DOCUMENT
%----------------------------------------------------------------------------------------

\begin{document} 

\pagestyle{empty} % Removes page numbers

{
 \raggedleft % Right-align all text
 \vspace*{\baselineskip} % Whitespace at the top of the page
 {\Large by Jo\~ao F. Matias Rodrigues and Christian von Mering\\
 Institute of Molecular Life Sciences\\University of Z\"urich\\Switzerland}\\[0.167\textheight] % Author name
 {\Huge HPC-CLUST v1.1.0 (June 2014)}\\[\baselineskip] % Main title which draws the focus of the reader
 {\Large \textit{Distributed hierarchical clustering of large number of pre-aligned sequences}}\par % Tagline or further description
}
\vfill % Whitespace between the title block and the publisher
\vspace*{3\baselineskip} % Whitespace at the bottom of the page

\newpage

\pagenumbering{Roman}
\tableofcontents
\newpage
\pagenumbering{arabic}

\section{Introduction}

HPC-CLUST is a set of tools designed to cluster large numbers (1 million or more)
pre-aligned nucleotide sequences. HPC-CLUST performs the clustering of sequences
using the Hierarchical Clustering Algorithm (HCA). There are currently three different
cluster metrics implemented: single-linkage, complete-linkage, and average-linkage.
There are currently 4 sequence distance functions implemented, these are:
- identity
    gap-gap counting as match
- nogap
    gap-gap being ignored
- nogap-single
    like nogap, but consecutive gap-nogap's count as a single mismatch
- tamura
    distance is calculated with the knowledge that transitions are more likely than
   transversions

One advantage that HCA has over other algorithms is that instead of producing only the
clustering at a given threshold, it produces the set of merges happening at each threshold.
With this approach, the clusters can be very quickly computed for every threshold with
little extra computations. This approach also allows the plotting of the variation of
number of clusters with clustering threshold without requiring the clustering to be run
for each threshold independently.

Another feature of the way HPC-CLUST is implemented is that the single, complete, and average
linkage clusterings can be computed in a single go with little overhead.


\section{INSTALLATION}

\subsection{Linux/Unix/MacOSX}

To install HPC-CLUST on a linux, unix, or MacOSX simply type:

./configure
make
make install

In the directory where you unpacked the package contents.
Alternatively, if you want the program to be installed to another
location instead of the default system wide /usr/local/ directory,
you can change the ./configure command to:

./configure --prefix=\$HOME/usr

This would install the program binaries to a directory usr/bin inside
your home directory (i.e.: \$HOME/usr/bin/hpc-clust), after you type
the command "make install".


\section{HPC-CLUST, HPC-CLUST-MPI}

Two programs are provided in the HPC-CLUST package.

When running on a single machine you should use HPC-CLUST. It is multi-threaded
so it can make use of several processors if you run it on a multi-core computer.

HPC-CLUST-MPI is the distributed computing version of
HPC-CLUST to be used on a cluster of computers. HPC-CLUST-MPI uses the MPI library
to automatically setup the communication between master and slave instances.


\section{USING HPC-CLUST}

To use HPC-CLUST you must first prepare your alignment file. You can accomplish this
using the INFERNAL aligner (http://infernal.janelia.org/), which aligns all sequences
against a model consensus sequence. Any other alignment such as multiple sequence
alignment will work as well, although, in practice, for very large numbers of sequences
it becomes difficult to compute a multiple sequence alignment.

HPC-CLUST takes the alignment file in the non-interleaved Stockholm format or aligned fasta format.
It detects automatically the format based on the first character. It switches to fasta format if the first
character is '\textgreater'.

An example of non-interleaved stockholm format is:\\
SEQUENCE\_ID1      ---ATGCAT---GCATGCATGC----... ...AT--AATT\\
SEQUENCE\_ID2      ---ATCCAC---GCACGCATGC-AT-... ...AT--ATAA\\
...\\
...

An example of aligned fasta format is:\\
{\textgreater}SEQUENCE\_ID1\\
---ATGCAT---GCATGCATGC----... ...AT--AATT\\
{\textgreater}SEQUENCE\_ID2\\
---ATCCAC---GCACGCATGC-AT-... ...AT--ATAA\\
...


An example of how to call the INFERNAL aligner such that it produces a file in the
right format is:
cmalign -1 --sub --matchonly -o alignedseqs.sto bacterial-ssu-model.cm unalignedseqs.fasta

This would produce a file with the aligned sequences "alignedseqs.sto", using the
"bacterial-ssu-model.cm" from the unaligned sequences file "unalignedseqs.fasta"

If the input sequences can be found in the file "alignedseqs.sto". Then to cluster the
sequences one simply has to run the following command:

hpc-clust -sl true alignedseqs.sto

This will cluster the sequences using single-linkage clustering. The output will be written
to the file "alignedseqs.sto.sl"

To run all linkage methods at once:

hpc-clust -sl true -cl true -al true alignedseqs.sto

In this case, the single, complete and average linkage clusterings will be written to the
"alignedseqs.sto.sl", "alignedseqs.sto.cl", and "alignedseqs.sto.al", respectively.

You can change the number of cpus that hpc-clust should use with the -ncpus \textless no\_cpus \textgreater argument.
The name of the output file using the -ofile \textless output\_filename \textgreater argument.
The threshold at which distances should be kept is specified using the -t \textless threshold \textgreater argument.
The lower this value, the more distances will be stored, and the lower the threshold until which
the clustering can proceed. However, the more distances are stored, the higher the memory 
requirement.



\section{USING HPC-CLUST-MPI}

To run the distributed version of HPC-CLUST, one needs to have the MPI library installed
before installing HPC-CLUST. If the MPI library is already set up, running HPC-CLUST-MPI
can be accomplished simply with:

mpirun -n 10 hpc-clust-mpi -sl true alignedseqs.sto

This will start 10 instances of hpc-clust-mpi, (1 master + 9 slaves) and perform single linkage
clustering on the "alignedseqs.sto" file. The results can be found in the "alignedseqs.sto.sl"
file as explained in the "USING HPC-CLUST" section above.

The MPI version of HPC-CLUST is best used in conjunction with a queuing system such as the
Sun Grid Engine system. In that case, if every instance of HPC-CLUST-MPI runs on a
separate machine it can make use of the combined memory of all the machines.

The exact command for submitting a job to the queuing system will depend on your local cluster
configuration. An example of the command to submit a job to run hpc-clust-mpi on 10 nodes would be:

qsub -b y -cwd -pe orte 10 mpirun -n 10 hpc-clust-mpi -sl true alignedseqs.sto


The speed of the computation and memory usage can be further optimized by having more than one
thread per node, in this manner the sequence data is shared between the threads. You can
increase the number of threads to use in each node by specifying the -nthreads \textless number\_threads \textgreater 
argument to hpc-clust-mpi.

\section{EXAMPLE MERGE FILE OUTPUT}

\# seqsfile: aligned-archaea-seqs.sto\\
\# OTU\_count Merge\_distance Merged\_OTU\_id1 Merged\_OTU\_id2\\
51094 1.0 37 38\\
51093 1.0 37 12176\\
51092 1.0 37 33214\\
51091 1.0 42 79\\
51090 1.0 42 2734\\
51089 1.0 42 2746\\
51088 1.0 42 2747\\
51087 1.0 42 2748\\
51086 1.0 42 2749\\
51085 1.0 42 2750\\
51084 1.0 42 2751\\
...

In the merge file output, each line indicates a merging event. For each merge event, the
number of OTUs existing after the merge, the identity distance at which the merge occurred,
and the OTU ids are shown. When a merge event occurs, the new OTU will have the same id 
as the smallest OTU id merged. For example, if OTUs' 37 and 12176 are merged, the new OTU
will have id 37.


\section{OBTAINING CLUSTERS FROM THE MERGE RESULTS}

Once the merge result files (.sl,.cl,.al) have been computed, you can produce the clusters
at different thresholds by using the script "make-otus.sh" provided.

make-otus.sh alignedseqs.sto alignedseqs.sto.sl 0.97

Would generate the clusters produced at the 97\% identity threshold from the single linkage
merge file.


\section{FAQ}
Frequently asked questions and other documentation can be found at
http://meringlab.org/software/hpc-clust/faq.html


\section*{ACKNOWLEDGMENTS}
Sebastian Schmidt




For further information contact: Jo\~ao F. Matias Rodrigues {\textless}joao.rodrigues@imls.uzh.ch\textgreater


\end{document}

